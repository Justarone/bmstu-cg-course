\chapter*{Заключение}
\addcontentsline{toc}{chapter}{Заключение}

В ходе курсового проекта было разработано программное обеспечение, которое предоставляет возможности загрузки параметров геометрической модели бицепса на узлах из конфигурационного файла, изменения этих параметров в интерактивном режиме, управления состоянием модели (сокращение и растяжение), а также положением (вращение, перемещение и масштабирование).
В процессе выполения данной работы были выполнены следующие задачи:
\begin{itemize}
    \item формально описана структура моделей мышцы и каркаса;
    \item рассчитаны формулы деформации геометрической модели с сохранением объема;
    \item выбраны алгоритмы трехмёрной графики, визуализирующие модель;
    \item реализованы алгоритмы для визуализации описанных выше объектов.
\end{itemize}

В процессе исследовательской работы было выяснено, что полученная геометрическая модель имеет ограничения, не позволяющие полностью воспроизвести поведение реального бицепса, однако реализованная модель способна получить схожее поведение: сохранить пропорции с уменьшенной в 5-6 раз амплитудой роста/уменьшения (из-за ограничения, связанного с постоянством объема), а также с требуемой степенью детализации получить некоторое статическое (изначальное) состояние.
