\chapter{Технологическая часть}

В данном разделе представлены средства разработки программного обеспечения, детали реализации и тестирование функций.

\section{Средства реализации}

В качестве языка программирования для разработки программного обеспечения был выбран язык программирования Rust\cite{rust}. Данный выбор обусловлен тем, что данный язык предоставляет весь требуемый функционал для решения поставленной задачи, а также обладает связанным с ним пакетным менеджером Cargo\cite{cargo}, который содержит инструменты для тестирования разрабатываемого ПО\cite{rusttest}.

Для создания пользовательского интерфейса программного обеспечения была использована библиотека gtk-rs\cite{gtk-rs}. Данная библиотека содержит в себе объекты, позволяющие напрямую работать с пикселями изображения, а также возможности создания панели управления с кнопками, что позволит в интерактивном режиме управлять изображением.

Для тестирования программного обеспечения были использованы инструменты пакетного менеджера Cargo\cite{cargo}, поставляемого вместе с компилятором языка при стандартном способе установке, описанном на официальном сайте языка\cite{rust}. 

В процессе разработки был использован инструмент RLS\cite{rls} (англ. \textit{Rust Language Server}), позволяющий форматировать исходные коды, а также в процессе их написания обнаружить наличие синтаксических ошибок и некоторых логических, таких как, например, нарушение правила владения\cite{rust-learn}.

В качестве среды разработки был выбран текстовый редактор VIM\cite{vim}, поддерживающий возможность установки плагинов\cite{vim-plugins}, в том числе для работы с RLS\cite{rls}.

\section{Реализация алгоритмов}

В листинге \ref{lst:muscle} представлена структура объекта мышцы, а также реализация методов деформации и триангуляции. В листинге \ref{lst:cg} представлена реализация алгоритмов компьютерной графики: $z$-буфера и Гуро.

\begin{lstinputlisting}[
        caption={Реализация объекта мышцы.},
        label={lst:muscle},
        style={rust},
        linerange={1-172,311-311}
    ]{../../src/lib/muscle.rs}
\end{lstinputlisting}

\begin{lstinputlisting}[
        caption={Реализация алгоритмов компьютерной графики.},
        label={lst:cg},
        style={rust},
        linerange={1-122}
    ]{../../src/lib/cg.rs}
\end{lstinputlisting}


\section*{Вывод}

В данном разделе были рассмотрены средства, с помощью которых было реализовано ПО, а также представлены листинги кода с реализацией объекта мышцы и алгоритмов компьютерной графики.
