\chapter*{Введение}
\addcontentsline{toc}{chapter}{Введение}

Наиболее удобным представлением информации для восприятия человеком является графическое представление. С развитием вычислительной техники, немаловажной функцией стала обработка информации, связанной с изображениями. Данная обработка подразумевает преобразование информации о некоторой местности в графическое представление этой местности и называется задачей \textit{синтеза изображения}.

В общем случае задача синтеза изображения трехмерных объектов представляет собой задачу имитации визуальной обстановки, т.е. искусственного построения изображений окружающей среды с такой степенью достоверности, которая достаточна для выработки и поддержания у пользователя программы навыков управления подвижными объектами. Таким образом, синтезируемое изображение должно быть динамическим и как можно более реалистичным.

В компьютерной графике решаются задачи разработки модели синтезируемой обстановки, задания исходных данных, связанных с определением положения наблюдателя и картинной плоскости, преобразования объектов сцены из одной системы координат в другую систему координат, отсечение объектов, вычисление перспективных проекций, удаление невидимых линий и поверхностей, создание реалистических изображений, растровой развертки простейших элементов изображения.

Цель данной работы - реализация ПО, которое предоставляет геометрическую модель человеческого бицепса на узлах, способную деформироваться (а именно сокращаться и растягиваться) с сохранением объема, а также предоставляет возможность деформации этой модели и применения афинного преобразования к ней.

Чтобы достигнуть поставленной цели, требуется решить следующие задачи:
\begin{itemize}
    \item формально описать структуру геометрической модели мышцы;
    \item рассчитать формулы, позволяющие деформировать геометрическую модель и обеспечивающие сохранение его объема;
    \item формально описать структуру геометрической модели каркаса мышцы;
    \item выбрать алгоритмы трехмёрной графики, позволяющие визуализировать модель;
    \item реализовать данные алгоритмы для визуализации описанных выше объектов.
\end{itemize}
