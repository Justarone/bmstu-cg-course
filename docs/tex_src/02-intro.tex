\chapter*{Введение}
\addcontentsline{toc}{chapter}{Введение}

Компьютерная графика – наука, занимающаяся созданием, визуализацией и обработкой изображений и элементов изображений посредством вычислительной техники. Задача компьютерной графики – установление связи между информацией неграфической природы и изображениями. Синтезом изображения называется преобразование формального описания местности в графическое представление.

Задача синтеза изображения трехмерных объектов представляет собой задачу имитации визуальной обстановки, т.е. искусственного построения изображений окружающей среды с такой степенью достоверности, которая достаточна для выработки и поддержания у пользователя программы навыков управления подвижными объектами. Таким образом, синтезируемое изображение стремятся сделать динамическим и реалистичным.

В компьютерной графике решаются задачи разработки модели синтезируемой обстановки, задания исходных данных, связанных с определением положения наблюдателя и картинной плоскости, преобразования объектов сцены из одной системы координат в другую, отсечение объектов, вычисление перспективных проекций и прочее.

Цель работы - разработать программное обеспечение, которое предоставляет возможности загрузки параметров модели геометрической модели бицепса на узлах из конфигурационного файла, изменения этих параметров в интерактивном режиме, управления состоянием модели (сокращение и растяжение), а также положением (вращение, перемещение и масштабирование).

Чтобы достигнуть поставленной цели, требуется решить следующие задачи:
\begin{itemize}
    \item формально описать структуру геометрической модели мышцы;
    \item рассчитать формулы деформации геометрической модели с сохранением объема;
    \item формально описать структуру геометрической модели каркаса мышцы;
    \item выбрать алгоритмы трехмёрной графики, визуализирующие модель;
    \item реализовать данные алгоритмы для визуализации описанных выше объектов.
\end{itemize}
