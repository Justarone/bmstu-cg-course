\chapter*{Введение}
\addcontentsline{toc}{chapter}{Введение}

Задача визуализации мышц решается в областях, где применяется компьютерная графика: разработка игр, профессиональное программное обеспечение, которое используется биологами и врачами, монтаж в кино и других. Такая задача решается геометрическими методами, физическими, а также методами, которые основаны на данных\cite{cgv}.

Метод решения задачи визуализации мышц выбирается исходя из требований и специфики предметной области, где эта задача ставится. Универсального метода решения подобной задачи нет: геометрические методы сохраняют инварианты за счёт решения соотношений без приближений, при этом редко находят применение в динамических средах; физические методы визуализации, наоборот, лучше визуализируют динамические средах, при этом аппроксимируя получающиеся во время решения алгебраические системы уравнений для ускорения вычислений или решая эти системы численными методами, что также приводит к потере точности. Методы, основанные на данных, применяются в случаях, когда требуется визуализировать мышцы существующего в нужном виде объекта.

Цель работы - разработать программное обеспечение, которое предоставляет возможности загрузки параметров модели геометрической модели бицепса на узлах из конфигурационного файла, изменения этих параметров в интерактивном режиме, управления состоянием модели (сокращение и растяжение), а также положением (вращение, перемещение и масштабирование).

Чтобы достигнуть поставленной цели, требуется решить следующие задачи:
\begin{itemize}
    \item формально описать структуру моделей мышцы и каркаса;
    \item рассчитать формулы деформации геометрической модели с сохранением объема;
    \item выбрать алгоритмы трехмёрной графики, визуализирующие модель;
    \item реализовать алгоритмы для визуализации описанных выше объектов.
\end{itemize}
